% Options for packages loaded elsewhere
\PassOptionsToPackage{unicode}{hyperref}
\PassOptionsToPackage{hyphens}{url}
%
\documentclass[
]{article}
\usepackage{amsmath,amssymb}
\usepackage{iftex}
\ifPDFTeX
  \usepackage[T1]{fontenc}
  \usepackage[utf8]{inputenc}
  \usepackage{textcomp} % provide euro and other symbols
\else % if luatex or xetex
  \usepackage{unicode-math} % this also loads fontspec
  \defaultfontfeatures{Scale=MatchLowercase}
  \defaultfontfeatures[\rmfamily]{Ligatures=TeX,Scale=1}
\fi
\usepackage{lmodern}
\ifPDFTeX\else
  % xetex/luatex font selection
\fi
% Use upquote if available, for straight quotes in verbatim environments
\IfFileExists{upquote.sty}{\usepackage{upquote}}{}
\IfFileExists{microtype.sty}{% use microtype if available
  \usepackage[]{microtype}
  \UseMicrotypeSet[protrusion]{basicmath} % disable protrusion for tt fonts
}{}
\makeatletter
\@ifundefined{KOMAClassName}{% if non-KOMA class
  \IfFileExists{parskip.sty}{%
    \usepackage{parskip}
  }{% else
    \setlength{\parindent}{0pt}
    \setlength{\parskip}{6pt plus 2pt minus 1pt}}
}{% if KOMA class
  \KOMAoptions{parskip=half}}
\makeatother
\usepackage{xcolor}
\usepackage[margin=1in]{geometry}
\usepackage{graphicx}
\makeatletter
\def\maxwidth{\ifdim\Gin@nat@width>\linewidth\linewidth\else\Gin@nat@width\fi}
\def\maxheight{\ifdim\Gin@nat@height>\textheight\textheight\else\Gin@nat@height\fi}
\makeatother
% Scale images if necessary, so that they will not overflow the page
% margins by default, and it is still possible to overwrite the defaults
% using explicit options in \includegraphics[width, height, ...]{}
\setkeys{Gin}{width=\maxwidth,height=\maxheight,keepaspectratio}
% Set default figure placement to htbp
\makeatletter
\def\fps@figure{htbp}
\makeatother
\setlength{\emergencystretch}{3em} % prevent overfull lines
\providecommand{\tightlist}{%
  \setlength{\itemsep}{0pt}\setlength{\parskip}{0pt}}
\setcounter{secnumdepth}{-\maxdimen} % remove section numbering
\usepackage{setspace}
\usepackage{ragged2e}
\justifying
\setlength{\parindent}{1em}
\setstretch{1.5}
\renewcommand{\normalsize}{\fontsize{12}{14}\selectfont}
\ifLuaTeX
  \usepackage{selnolig}  % disable illegal ligatures
\fi
\usepackage{bookmark}
\IfFileExists{xurl.sty}{\usepackage{xurl}}{} % add URL line breaks if available
\urlstyle{same}
\hypersetup{
  pdftitle={Aplicação de Técnicas de Otimização dentro da Logistica Humanitária: Um estudo de caso},
  pdfauthor={Edison Fernando da Silva Nunes},
  hidelinks,
  pdfcreator={LaTeX via pandoc}}

\title{Aplicação de Técnicas de Otimização dentro da Logistica
Humanitária: Um estudo de caso}
\author{Edison Fernando da Silva Nunes}
\date{2024-09-07}

\begin{document}
\maketitle

\textbf{1. Introdução}

A logística humanitária refere-se à organização e coordenação das
atividades logísticas envolvidas em operações de ajuda humanitária.
Trata-se de um campo especializado que lida com os desafios enfrentados
pelos trabalhadores humanitários, como a entrega rápida de suprimentos
essenciais em áreas remotas ou afetadas por desastres. Os desastres
resultam de eventos adversos, naturais ou provocados pelo homem, que
impactam ecossistemas vulneráveis, causando danos humanos, materiais
e/ou ambientais, além de prejuízos econômicos e sociais.
Especificamente, os desastres naturais têm consequências drásticas para
sociedades, regiões e países, resultando em vítimas fatais, feridos,
desabrigados, cidades destruídas e altos custos de reconstrução. No
Brasil, um levantamento realizado pelo Centro de Estudos e Pesquisas
sobre Desastres (CEPED, 2013) no período de 1991 a 2012 registrou 38.996
desastres naturais, com destaque para os anos de 2010, 2011 e 2012, que
concentraram 78\% dos desastres registrados. Anualmente, centenas de
desastres naturais, como terremotos, enchentes e inundações, ocorrem em
todo o mundo. Diversas atividades de logística humanitária, como o
transporte de suprimentos, gestão de estoques, tecnologia da informação
e comunicação, distribuição eficiente de alimentos, água e abrigos, bem
como serviços de saúde e gestão de resíduos, são necessárias. Também
pode incluir outras organizações e agências governamentais para garantir
uma resposta eficaz. A logística humanitária enfrenta muitos desafios,
como a falta de infraestrutura básica em áreas afetadas, condições
climáticas e ambientais adversas, logística precária e questões de
segurança. Além disso, há dificuldades em garantir equidade e acesso
igualitário aos recursos entre as populações afetadas. Nesse contexto, o
desastre natural ocorrido durante o mês de maio de 2024 na região sul do
Brasil, no estado do Rio Grande do Sul, devido a chuvas intensas e
consequentes inundações, afetou gravemente as regiões dos vales dos rios
Taquari, Caí, Pardo, Jacuí, Sinos e Gravataí, além de Porto Alegre pelo
rio Guaíba e as regiões de Pelotas e Rio Grande pela Lagoa dos Patos. As
inundações devastaram áreas urbanas e rurais, resultando em danos
significativos, como desabrigados e desalojados, além de grande prejuízo
material. Ao todo, 478 municípios foram afetados, com uma população de
2.398.255 pessoas impactadas. As inundações resultaram em 806 feridos,
31 desaparecidos e 182 óbitos confirmados. A partir da coleta de dados
de monitoramento já realizados e pela produção prévia de mapas do
município de Rio Grande-RS, que indicavam os locais possivelmente
afetados pela elevação das águas da Lagoa dos Patos, foi possível
tomadas de decisões mais assertivas e maior previsibilidade das ações.
Ao longo dos anos, a logística humanitária avançou consideravelmente,
com o uso de tecnologias avançadas, como drones e análise de dados, para
melhorar a eficiência e eficácia das operações. Também foi desenvolvida
uma rede global de organizações e agências especializadas em logística
humanitária para o compartilhamento de conhecimentos e recursos. A
logística humanitária desempenha um papel vital na resposta a
emergências e desastres, garantindo que os recursos essenciais cheguem
às pessoas que mais precisam. Além disso, desempenha um papel importante
na recuperação e reconstrução pós-desastre, ajudando a restabelecer
infraestruturas básicas e a fornecer apoio contínuo às comunidades
afetadas.

\textbf{2. Justificativa}

A crescente frequência e severidade dos desastres naturais, como as
inundações recentes no Rio Grande do Sul em maio de 2024, evidenciam a
necessidade urgente de desenvolver e implementar estratégias eficazes de
logística humanitária. Esses eventos sublinham a importância de uma
resposta rápida e coordenada para salvar vidas, minimizar danos e apoiar
a recuperação das comunidades afetadas. No entanto, os desafios impostos
por infraestrutura precária e condições adversas frequentemente
dificultam a entrega eficaz de ajuda humanitária. A aplicação de
técnicas de otimização de fluxo de redes em operações de socorro pode
aprimorar significativamente a distribuição de recursos e serviços de
emergência. Essas técnicas não apenas melhoram a alocação eficiente de
suprimentos críticos, como também reduzem os custos operacionais
envolvidos, garantindo que os recursos limitados sejam utilizados da
melhor forma possível. Tecnologias avançadas e uma colaboração estreita
entre organizações são essenciais para enfrentar os desafios logísticos
em situações de desastre, aumentando a eficácia das operações de
emergência. Este estudo visa fornecer ferramentas e conhecimentos
práticos para gestores e tomadores de decisão, promovendo maior
resiliência e capacidade de resposta em situações de emergência. Ao
implementar técnicas de otimização e estratégias logísticas avançadas,
espera-se melhorar a prontidão e a eficiência das respostas a desastres,
fortalecendo a capacidade das comunidades de lidar com os impactos de
eventos extremos e facilitando a recuperação pós-desastre.

\textbf{3. Objetivos}

\textbf{3.1. Objetivo Geral}

Aplicar técnicas de otimização de fluxo de redes para minimizar o tempo
de atendimento às vítimas de desastres de inundações para o município de
Rio Grande - RS, melhorando a eficiência na distribuição de recursos e
na alocação de serviços de emergência.

\textbf{3.2. Objetivos Específicos}

\begin{enumerate}
\def\labelenumi{\arabic{enumi}.}
\item
  Mapear a região identificando locais com menor e maior risco de
  inundações;
\item
  Desenvolver um modelo de otimização para a localização de abrigos para
  vítimas de desastres provocados por inundações no município de Rio
  Grande - RS;
\item
  Desenvolver um modelo de roteamento para a entrega dos principais
  suprimentos (alimentos, remédios e roupas);
\item
  Realizar simulações para os dois modelos, apontando a solução ótima;
\item
  Propor recomendações para estratégias de logística humanitária
  baseadas nos resultados obtidos.
\end{enumerate}

\textbf{4. Metodologia}

O trabalho proposto, de acordo com Pizzolato (2012), seguirá a
metodologia básica da Pesquisa Operacional, passando pela identificação
do problema, a formulação de um modelo matemático utilizando hipóteses
simplificadoras, a resolução do modelo, a validação dos resultados e,
posteriormente, o oferecimento de propostas para implementação.

\textbf{4.1. Coleta de Dados}

Dados meteorológicos, relatórios dos últimos desastres e informações
sobre a infraestrutura do município serão coletados para análise.

\textbf{4.1.1. O modelo da P-Mediana}

Para o número ótimo e localização dos abrigos, pretende-se utilizar o
modelo das P-Medianas. Nesse modelo, busca-se minimizar a distância
média, determinando a localização ideal de (p) instalações (abrigos) de
forma que a demanda total da área seja atendida de maneira eficiente.
Esses problemas são comumente aplicados em contextos de planejamento
urbano, logística de distribuição, e situações de resposta a
emergências, onde a meta é reduzir o tempo ou o custo de transporte
entre os centros de distribuição e os abrigos. A solução do problema
envolve escolher (p) locais de um conjunto de possíveis locais de
abrigos para que a soma das distâncias ponderadas (baseadas na demanda)
dos centros de distribuição até as instalações mais próximas seja
minimizada. Sejam N=\{1,\ldots,n\} o conjunto de pontos de
demanda;\(i \in \mathbb{N}\), um determinado cliente ou
vértice;\(j \in \mathbb{N}\) uma instalação em potencial ou mediana; p o
número de instalações de serviço ou medianas a serem localizadas;
\(w_i\) o peso ou importância de cliente
\emph{i};\([d_{ij}]_{n \times n}\) a matriz simétrica de distâncias de
cada cliente \emph{i} à instalação \emph{j} , com
\(d_{ii}=0, \forall i; [x_{ij}]_{n \times n}\) a matriz de alocação de
cada cliente \emph{i}; onde \(x_{ij}=1\) se o cliente \emph{i} é alocado
à instalação \emph{j} e \(x_{ij}=0\) , caso contrário; \(x_{jj}=1\)
indica que \emph{j} é uma mediana e \(x_{jj}=0\) em caso contrário.
Então, o modelo da p-mediana é apresentado da seguinte forma:

\[
Min \quad  z=\sum_{i \in \mathbb{N}} \sum_{j \in \mathbb{N}} w_{i}d_{ij}x_{ij}    (1)
\] Sujeito a

\[
\sum_{j \in \mathbb{N}}x_{ij}=1; i \in \mathbb{N}
\]

\[
\sum_{j \in \mathbb{N}}x_{jj}=p
\]

\[
x_{ij} \leq x_{jj}; i,j \in \mathbb{N}
\]

\[
x_{ij} \in \mathbb{0,1}; i,j \in \mathbb{N}\] onde a função objetivo (1)
indica a minimização das distâncias ponderadas entre os clientes e os
opostos que oferecem serviços; as restrições em (2) indicam que cada
cliente \emph{i} é alocado a somente uma instalação \emph{j} ; a
restrição (3) garante que somente \emph{p} instalações oferecem o
serviço proposto; as restrições em (4) afirmam que um cliente somente é
atendido num local onde existe uma instalação que oferece o serviço, e
as restrições em (5) impõem variáveis de decisão binárias. Cabe lembrar
que pode existir uma lista prévia de pontos candidatos a serem
escolhidos como mediana; nesse caso o modelo acima sofre pequenas
modificações e recebe o nome tradicional de problema de localização de
uma planta simples ou em inglês \emph{simple plant location model
(SPLP)} Observamos ainda que, em lugar de localização, o modelo acima
pode ser interpretado como modelo de zoneamento, em que se busca dividir
o espaço em \emph{p} zonas. Nessa ótica de zoneamento, o modelo de
p-mediana pode ser aplicadoao problema de classificação de um conjunto
de padrões, conhecido como \emph{cluster} \emph{analysis} (Xavier e
Xavier, 2011), que busca agrupar elementos com padrões semelhantes em
certo \emph{cluster} e com padrões diferentes em \emph{clusters}
distintos.

\textbf{4.2. Desenvolvimento do Modelo}

Serão utilizados modelos matemáticos para a localização dos abrigos e
para a entrega de suprimentos. Estes modelos serão alimentado com
variáveis como: quantidades de suprimentos, veículos disponíveis, rotas
de transporte e demanda por serviços de emergência.

\textbf{4.2.2. O Modelo para o Roteamento de Veículos}

Os problemas de roteamento de veículos (PRV) consistem basicamente em
determinar rotas para realizar algum tipo de serviço, de maneira que o
custo seja mínimo. Resolver um PRV significa procurar a forma de
distribuir a um ou mais veículos uma determinada lista de compromissos
de entrega, associados a determinados pontos, devendo os veículos
retornar ao ponto de origem ao final do trabalho. A formulação
matemática para um Problema de Roteamento de Veículos (PRV) é
apresentada abaixo:

Minimizar , \[z=\sum_{i,j}(c_{ij}\sum_{k}x_{ijk})\] Sujeito a:
\[\sum_{k}y_{ik}=1 \quad \text{para} \quad  i=2, \dots,n\]

\[\sum_{k}y_{ik}=1\quad \text{para} \quad  i=1\]

\[\sum_{i}q_{i}y_{ik} \leq Q_{k} \quad \text{para} \quad  k=1, \dots,m\]
\[\sum_{j}x_{ijk} \leq\sum x_{jix}=y_{ik} \quad \text{para}\quad i=1,\dots,n\quad k=1,\dots,m\]
\[y_{ik} \in \{0,1\}\quad\text{para}\quad i=1,\dots,n\quad\text{para}\quad k=1,\dots,m\]
\[x_{ijk} \in \{0,1\}\quad\text{para}\quad i,j=1,\dots,n\quad\text{para}\quad k=1,\dots,m\]

Onde:

Seja 𝑁=\{1,\ldots,𝑛\} N=\{1,\ldots,n\} o conjunto de pontos de demanda,
onde:

𝑖∈𝑁i∈N representa um ponto de demanda (cliente ou vértice);

𝑗∈𝑁j∈N é um local potencial para um abrigo; \(d_{ij}\) é a distância
entre o ponto de demanda \emph{i} e o local \emph{j}; \(x_{j}\) é uma
variável binária que indica se um abrigo será instalado no local
\emph{j} (1 se sim, 0 caso contrário); \(y_{ij}\) é uma variável binária
que indica se o ponto \emph{i} será atendido pelo abrigo \emph{j}.

\textbf{4.3. Simulação e Análise}

Os modelos serão simulados usando dados reais das inundações de 2024 no
município de Rio Grande - RS, focando na eficiência de distribuição e na
redução de custos operacionais. Testes de sensibilidade serão realizados
para avaliar diferentes cenários.

\textbf{5. Resultados Esperados}

Melhoria na alocação de abrigos: Espera-se que o modelo de P-Mediana
sugira um conjunto de locais estratégicos para a instalação de abrigos,
considerando a minimização da distância média entre os pontos de demanda
e os abrigos. Isso deve reduzir significativamente o tempo de resposta
durante desastres, permitindo que as vítimas sejam acolhidas de forma
mais rápida e eficiente. A localização estratégica dos abrigos é crucial
em cenários de inundações, onde o tempo de resposta pode ser
determinante para salvar vidas e mitigar o impacto dos desastres. Além
disso, a distribuição eficiente dos abrigos contribui para a otimização
de recursos, uma vez que evita sobrecarga em determinados locais e
garante que todos os abrigos recebam um fluxo equilibrado de vítimas.

Este projeto busca contribuir para a logística humanitária ao aplicar
modelos de P-Medianas e técnicas de roteamento, focando na diminuição
dos tempos de atendimento às vítimas de inundações no município de Rio
Grande -- RS. A aplicação desses modelos não só auxilia na escolha do
local mais próximo que servirá de abrigo, mas também no abastecimento
adequado de cada abrigo com suprimentos essenciais, como alimentos,
medicamentos e água. Além disso, espera-se que o estudo possa oferecer
insights valiosos sobre como aprimorar a alocação de recursos em
cenários futuros, utilizando técnicas de otimização para criar uma rede
de resposta rápida e eficiente. A utilização de modelos matemáticos e
algoritmos de roteamento também permitirá identificar as melhores rotas
de distribuição de suprimentos, minimizando o tempo de transporte e os
custos operacionais envolvidos, resultando em um sistema de resposta
mais ágil e menos oneroso.

\textbf{6. Cronograma}

\begin{table}[ht]
\centering
\begin{tabular}{|p{3cm}|p{1.5cm}|p{1.5cm}|p{1.5cm}|p{1.5cm}|p{1.5cm}|p{1.5cm}|p{1.5cm}|p{1.5cm}|}
\hline
Etapas & 1º trim 2024 & 2º trim 2024 & 3º trim 2024 & 4º trim 2024 & 1º trim 2025 & 2º trim 2025 & 3º trim 2025 & 4º trim 2025 \\ \hline
Revisão Bibliográfica   &    &    &    &    &    &    &    &    \\ \hline
Submissão de artigo da
disciplina de Seminários
em Ambientometria   &    &    &    &    &    &    &    &    \\ \hline
Estudo dos Modelos   &    &    &    &    &    &    &    &    \\ \hline
Desenvolvimento dos Modelos &    &    &    &    &    &    &    &    \\ \hline
Conclusões   &    &    &    &    &    &    &    &    \\ \hline
\end{tabular}
\caption{Sua tabela de 9 colunas e 6 linhas}
\end{table}

\newpage

\textbf{Referências Bibliográficas}

Pizzolato, N.D. (2012). Introdução à Pesquisa Operacional: conceitos e
práticas. LTC.

Xavier, M.C., \& Xavier, R.L. (2011). Problemas de localização e
roteamento. Elsevier.

Centro de estudos e Pesquisas sobre Desastres (CEPED). (2013). Desastres
Naturais no Brasil 1992-2012. CEPED.

Novaes, A.G. Logística e gerenviamento da cadeia de distribuição:
estratégia, operação e avaliação. Rio de Janeiro: Campus-Elsevier, 2004.

Leiras, A.; Yoshizaki, H.T.Y.; Samed, M.M.A.; Gonçalves, M.B.(2017).
Logística Humanitária. 1.ed.Rio de Janeiro: Elsevier, v.1.

Okanta, S.D. \& Olaomi, J.,2023 Aplicando técnicas de otimização de
fluxo de rede para minimizar custos associados a desastres de inundação,
Jambá: Journal of Disaster Risk Studies 15(1).

Baraka J.C.M.; Yadavalli V.S.S.; Singh R. Um modelo de transporte para
uma operação eficaz de ajuda em desastres na SADC. Journal Sul-Africano
de Engenharia Industrial, agosto de 2017, Vol 28(2), pp 46-58.

\end{document}
